\RequirePackage{plautopatch}
% 一応pLaTeX用パッチを当てておく
% クラスオプションでdvipdfmxを読み込めばusepackageで個別に指定する必要がなくなる
% jarticleは非推奨なので使いたくないが、レギュレーションなので大変遺憾に思いつつ従う
\documentclass[dvipdfmx,12pt,a4j]{jarticle}
% dvipdfmxはグローバルオプションとして読み込んでいるので個別のパッケージのオプションとしては不要
\usepackage{graphicx}
\usepackage{amsmath,amsbsy}
% ロゴ関係
\usepackage{bxtexlogo}
\bxtexlogoimport{*,**}
% Hオプションを使いたいので読み込む
\usepackage{here}
% ソースコードを表示するのに読み込む
\usepackage{listings}
% シンタックスハイライトのため
\usepackage{xcolor}
% url挿入のため
\usepackage{url}
% 枠のため
\usepackage{ascmac}
% xcolorの色定義
\definecolor{solarized@base03}{HTML}{002B36}
\definecolor{solarized@base02}{HTML}{073642}
\definecolor{solarized@base01}{HTML}{586e75}
\definecolor{solarized@base00}{HTML}{657b83}
\definecolor{solarized@base0}{HTML}{839496}
\definecolor{solarized@base1}{HTML}{93a1a1}
\definecolor{solarized@base2}{HTML}{EEE8D5}
\definecolor{solarized@base3}{HTML}{FDF6E3}
\definecolor{solarized@yellow}{HTML}{B58900}
\definecolor{solarized@orange}{HTML}{CB4B16}
\definecolor{solarized@red}{HTML}{DC322F}
\definecolor{solarized@magenta}{HTML}{D33682}
\definecolor{solarized@violet}{HTML}{6C71C4}
\definecolor{solarized@blue}{HTML}{268BD2}
\definecolor{solarized@cyan}{HTML}{2AA198}
\definecolor{solarized@green}{HTML}{859900}
% listingsのスタイル定義
\lstdefinestyle{c}{
  language=c,
  numbers=left,
}
\lstset{
basicstyle=\small\ttfamily\color{solarized@base00},
rulesepcolor=\color{solarized@base03},
numberstyle=\scriptsize\color{solarized@base01},
keywordstyle=\color{solarized@blue},
stringstyle=\color{solarized@cyan}\ttfamily,
commentstyle=\color{solarized@base01},
emphstyle=\color{solarized@red},
backgroundcolor=\color{solarized@base3},
sensitive=true,
breaklines=true,
breakatwhitespace=true,
framerule=0pt,
frame=l,
showstringspaces=false,
tabsize=2,
basewidth={0.57em,0.52em},
}
\title{プログラミング 第4回レポート}
\author{202111609 仲村和士}
\date{\today}
\begin{document}
\maketitle
\section{はじめに}
今回は連結リストのデータ構造に関する問題である。連結リストは個人的にはJavaのCollectionとして実装されているものが一番最初に浮かぶが、実際に実装してみると、オブジェクト指向ではないC言語ではJavaに比べて少々扱いにくい印象であった。今回は授業資料のほかに、C++で解説された書籍~\cite{algorithm} で全体的な知識を得た。日本語の書誌情報を含むのでpbibtexを推奨する。

\section{設問(1)}
\subsection{課題内容と方針}
双方向連結リストを扱いやすい関数にまとめるほか、末尾から辿って出力する関数を作る問題である。

方針としては、講義ノート(14)の図9より図11のほうが作成したい体裁に近いことから、図11を双方向リストに修正する方向で実装する。修正する部分は図9の解説を読めば十分である。

\subsection{実装}
図11に対する変更点を簡単に説明する。
まず8行目で前のElementへの参照を持つように変更している。
次に、27行目ではnext同様に初期化している。53行目ではappendする際に戻り方向の参照も設定するようにしている。

66行目からのrprintAllElement関数では、最後の要素l-\verb+>+tから逆順にhにたどり着くまで要素の値を出力している。
mainでは最後にこの関数を呼んでいる。

\lstinputlisting[caption=s2111609-1.c,style=c]{./src-c/s2111609-1.c}

\subsection{確認}
この設問の要求は、双方向連結リストの関数化、rprintAllElements関数の実装、標準入力の値をリストに入れて出力することである。
もともと単方向のリストが実装されていたので、rprintAllElements関数で正しく逆順に辿れていることが確認できればよい。また、要素数が0のコーナーケース(となりうる状況)も一応確認する。以下の実行例から、正しく実装されていることが確認できる。

\begin{itembox}[l]{実行例}
\begin{verbatim}
  $ gcc s2111609-1.c
  $ ./a.out
  input a number (quit when 0): 1
  input a number (quit when 0): 4
  input a number (quit when 0): 8
  input a number (quit when 0): 5
  input a number (quit when 0): 1
  input a number (quit when 0): 67
  input a number (quit when 0): 356
  input a number (quit when 0): 0
  val = 1
  val = 4
  val = 8
  val = 5
  val = 1
  val = 67
  val = 356
  reverse
  val = 356
  val = 67
  val = 1
  val = 5
  val = 8
  val = 4
  val = 1
  $ ./a.out 
  input a number (quit when 0):  0
  reverse
\end{verbatim}
\end{itembox}

\subsection{難しかった点など}
LinkedList自体がそれなりに難しいとは思うが、事前にしっかり理解してから注意深く実装したので問題なかった。

\section{設問(2)}
\subsection{課題内容と方針}
受け取った入力を昇順になるように挿入するinsertElement関数を作る課題である。

方針としては、関数を分割する。すなわち、特定のElementのあとにElementを挿入する(ポインタを繋ぎ変える)関数を別に作っておき、挿入する場所を探して挿入する。挿入する場所を探す際、それまでのノードはすでに昇順になっているという条件を用いれば、前から順番にリストの要素と挿入する要素の値を比較すればよい。

\subsection{実装}
前問との変更点は57行目からのinsert関数とinsertElement関数を追加した点である。insert関数は引数に2つのElementのポインタp, vをとり、pのあとにvを挿入する関数である。ここでは、pが末尾の要素の場合にp-\verb+>+nextがNULLになることに注意する。60行目のifではNULL値の場合に61行目が問題になるのでそれを場合分けしている。67行目のifではpが末尾の要素であったときにvを末尾に更新する処理である。

insertElement関数では、新しい要素を挿入するべき場所をループで探してinsert関数を用いて挿入する。hを起点として順番に探して行く構造である。79行目からの判定について説明すると、最初のifでは、今見ている要素seekの次の要素がNULLであるときにはseekは末尾の要素であるから、最後に挿入する。else ifでは次の要素の値が、挿入する値よりも大きければ、現在見ているseek要素の次に挿入する。これらの挿入処理はappendする関数とinsert関数で分けることもできるが、tの更新処理まで行っているinsertが万能なので結局同じように書ける。\footnote{orで繋いで一つのifで書くこともできるが、内容は同じなので好みの問題だろう。}
それ以外のときにはまだ挿入する場所が見つかっていないのでseekを次の要素にする。これを繰り返して正しい場所に挿入する。

main関数ではそのままappendしていくlist1と昇順になっているlist2を用意して比較できるようにした。


\lstinputlisting[caption=s2111609-2.c,style=c]{./src-c/s2111609-2.c}

\subsection{確認}
重複要素や負数を含んだ入力に対して正しく対応できればよい。また、処理冒頭で要素数が0のときが必ずあるのであえて分けて検証する必要はあまりない。

\begin{itembox}[l]{実行例}
\begin{verbatim}
  $ gcc s2111609-2.c -std=c89
  $ ./a.out
  input a number (quit when 0): -5
  input a number (quit when 0): -100
  input a number (quit when 0): -10
  input a number (quit when 0): -5
  input a number (quit when 0): 7
  input a number (quit when 0): -3
  input a number (quit when 0): 100
  input a number (quit when 0): 7
  input a number (quit when 0): 0
  val = -5
  val = -100
  val = -10
  val = -5
  val = 7
  val = -3
  val = 100
  val = 7
  sorted
  val = -100
  val = -10
  val = -5
  val = -5
  val = -3
  val = 7
  val = 7
  val = 100
\end{verbatim}
\end{itembox}

\subsection{難しかった点など}
当初、NULL値の場合分けに気づかず、Segmentation faultを起こしたが、割とすぐに発見できた。


\section{設問(3)}
\subsection{課題内容と方針}
リストを文字列に対応させて入出力(ただし、改行文字は削除する)を行う課題である。

基本的に指示通りの実装をすればよいが、文字列の受け取りは前回の課題を参考にする。

\subsection{実装}
まず、前問までの実装で本問には必要のない関数は削除した。
また、int型を入力していた部分をchar*型に変更した。
printAllElements関数では、\%dを\%sに変更して文字列として表示するように変更した。
69行目からは改行文字を除外するのに必要なchomp関数を講義ノート(14)をもとに記述した。main関数ではbufで文字列を受け取り、その後に動的なメモリにコピーし、そのポインタをリストで管理している。
\lstinputlisting[caption=s2111609-3.c,style=c]{./src-c/s2111609-3.c}

\subsection{確認}
動的なメモリ割り当てと、改行文字の削除は実装している。実行例では入力内容が1行ずつ表示されることを確認すれば要件を満たす。



\begin{itembox}[l]{実行例}
\begin{verbatim}
  $ gcc s2111609-3.c -std=c89
  $ ./a.out
  abcd
  1234
  a
  12334556
  welcome
  -----list content-----
  abcd
  1234
  a
  12334556
  welcome
\end{verbatim}
\end{itembox}

\subsection{難しかった点など}
特になかった。

\section{感想}
これは副次的なものではあるが、LinkedListの実装を通して、newキーワードの振る舞いやJavaのArrayListとLinkedListの違いを正しく理解することができた。また、設問(3)では、C89では使えないがgenericsの威力が実感できた。



\bibliographystyle{junsrt}
\bibliography{cite}
\end{document}