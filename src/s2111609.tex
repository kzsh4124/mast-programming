\RequirePackage{plautopatch}
% 一応pLaTeX用パッチを当てておく
% クラスオプションでdvipdfmxを読み込めばusepackageで個別に指定する必要がなくなる
% jarticleは非推奨なので使いたくないが、レギュレーションなので大変遺憾に思いつつ従う
\documentclass[dvipdfmx,twocolumn,12pt]{jarticle}
% dvipdfmxはグローバルオプションとして読み込んでいるので不要
\usepackage{graphicx}
% ロゴ関係
\usepackage{hologo}
\usepackage{bxtexlogo}
\bxtexlogoimport{*,**}
% Hオプションを使いたいので読み込む
\usepackage{here}
% ソースコードを表示するのに読み込む
\usepackage{listing}
% シンタックスハイライトのため
\usepackage{xcolor}
% 枠かこみのため
\usepackage{tcolorbox}
\title{プログラミング 第1回レポート}
\author{202111609 仲村和士}
\date{\today}

\begin{document}
\maketitle
\section{はじめに}
課題内容を始める前に重要なことを何点かこの節で述べる。

\subsection{\BibTeX のコンパイルについて(重要)}
はじめに本レポートをコンパイルする際の注意点について述べたい。本レポートの文献情報には日本語書籍が含まれている。したがって通常の\BibTeX ではなく\pBibTeX を利用することを推奨する。
\begin{verbatim}
  $ pbibtex -kanji=utf-8 file_name
\end{verbatim}

\subsection{\LaTeX 利用の方針}
随時インターネットで調べた結果を活用するが、日本語\LaTeX の代表的書籍である、奥村、黒木~\cite{bibunsho}に記述があるものはこれに従う。講義資料と齟齬がある点についてはその都度記述する。

ただし、例外的にjarticleを利用する点のみ前述の書籍の記述から外れるものとする。\LaTeX の作法としては非推奨であるが、今回のレポートのレギュレーションである以上それに従うものとする。

\section{Linuxのコマンド利用}

\section{ソースコードの挿入}
\subsection{課題内容と方針}

\section{本の紹介}
イントロ1\verb+~+2文述べる。
\begin{itemize}
  \item 本1 \cite{bibunsho}
  \item 本2
  \item 本3
\end{itemize}

\bibliographystyle{jplain}
\bibliography{citation}
\end{document}
